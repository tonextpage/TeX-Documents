\documentclass{article}
\usepackage[english]{babel}
\usepackage[utf8]{inputenc}
\usepackage[a4paper]{geometry}
\usepackage{mathtools}
\usepackage{amsthm, amsmath, amsfonts, amssymb}
\usepackage[]{mdframed}
\usepackage{hyperref}
\usepackage{multicol}
\usepackage[shortlabels]{enumitem}
\usepackage{float}
\usepackage{booktabs}
\usepackage[dvipsnames]{xcolor}
\usepackage{graphicx}
\usepackage{wrapfig}
\usepackage{tikz}
\usepackage{tikz-cd}
\usepackage[skins, vignette, raster, theorems, breakable]{tcolorbox}
\usepackage{pgfkeys}
\newcommand*\circled[1]{\tikz[baseline=(char.base)]{\node[shape=circle,draw,inner sep=1pt] (char) {#1};}}
\usepackage{textcomp, gensymb}
\usepackage{kotex}
\usepackage{setspace}
\usepackage{cancel}
\usepackage{diagbox}
\usepackage[no-math]{fontspec}

%%%%% font setting %%%%%%%%%%%%%%%%%%%%%%%%%%%%%%%%%%%%%%%%%%%%%%%%%%%%%%
\setmainfont{KoPub Dotum}[
	Extension = .ttf,
	UprightFont = * Light,
	BoldFont = * Bold,
	ItalicFont = * Light,
]
%%%%%%%%%%%%%%%%%%%%%%%%%%%%%%%%%%%%%%%%%%%%%%%%%%%%%%%%%%%%%%%%%%%%%%%%%

%%%%% author & date %%%%%%%%%%%%%%%%%%%%%%%%%%%%%%%%%%%%%%%%%%%%%%%%%%%%%
\author{}
\date{}
%%%%%%%%%%%%%%%%%%%%%%%%%%%%%%%%%%%%%%%%%%%%%%%%%%%%%%%%%%%%%%%%%%%%%%%%%

%%%%% TCBox - Definition %%%%%%%%%%%%%%%%%%%%%%%%%%%%%%%%%%%%%%%%%%%%%%%%
% usage:
%	\begin{dfn}{Definition Title}{Reference Keyword}
%		Text Here
%	\end{dfn}
% reference:
%	\ref{dfn:Reference Keyword}
%%%%%%%%%%%%%%%%%%%%%%%%%%%%%%%%%%%%%%%%%%%%%%%%%%%%%%%%%%%%%%%%%%%%%%%%%
\DeclareTcbTheorem[number within=section]{dfn}{정의}{
	enhanced, breakable, left=4mm, right=4mm, top=4mm, bottom=2mm,
	colback=Mahogany!5, coltitle=white, colframe=Mahogany,
	before title={\,}, title={#1}, after title={\,}, fonttitle=\bfseries,
	attach boxed title to top left={xshift=4mm, yshift=-3.45mm},
	boxed title style={colback=Mahogany, left=0mm,right=0mm, top=0mm, bottom=0mm},
	separator sign none, description delimiters parenthesis
}{dfn}

%%%%% TCBox - Theorem %%%%%%%%%%%%%%%%%%%%%%%%%%%%%%%%%%%%%%%%%%%%%%%%%%%
% usage:
%	\begin{thm}{Theorem Title}{Reference Keyword}
%		Text Here
%	\end{thm}
% reference:
%	\ref{thm:Reference Keyword}
%%%%%%%%%%%%%%%%%%%%%%%%%%%%%%%%%%%%%%%%%%%%%%%%%%%%%%%%%%%%%%%%%%%%%%%%%
\DeclareTcbTheorem[use counter from=dfn]{thm}{정리}{
	enhanced, breakable, left=4mm, right=4mm, top=4mm, bottom=2mm,
	colback=BurntOrange!5, coltitle=white, colframe=BurntOrange,
	before title={\,}, title={#1}, after title={\,}, fonttitle=\bfseries,
	attach boxed title to top left={xshift=4mm, yshift=-3.45mm},
	boxed title style={colback=BurntOrange,left=0mm,right=0mm, top=0mm, bottom=0mm,},
	separator sign none, description delimiters parenthesis
}{thm}

%%%%% TCBox - Lemma %%%%%%%%%%%%%%%%%%%%%%%%%%%%%%%%%%%%%%%%%%%%%%%%%%%%%
% usage:
%	\begin{lemma}{Lemma Title}{Reference Keyword}
%		Text Here
%	\end{lemma}
% reference:
%	\ref{lemma:Reference Keyword}
%%%%%%%%%%%%%%%%%%%%%%%%%%%%%%%%%%%%%%%%%%%%%%%%%%%%%%%%%%%%%%%%%%%%%%%%%
\DeclareTcbTheorem[use counter from=dfn]{lemma}{보조정리}{
	enhanced, breakable, left=4mm, right=4mm, top=4mm, bottom=2mm,
	colback=BurntOrange!5, coltitle=white, colframe=BurntOrange,
	before title={\,}, title={#1}, after title={\,}, fonttitle=\bfseries,
	attach boxed title to top left={xshift=4mm, yshift=-3.45mm},
	boxed title style={colback=BurntOrange, left=0mm,right=0mm, top=0mm, bottom=0mm},
	separator sign none, description delimiters parenthesis
}{lemma}

%%%%% TCBox - Proposition %%%%%%%%%%%%%%%%%%%%%%%%%%%%%%%%%%%%%%%%%%%%%%%%
% usage:
%	\begin{prop}{Proposition Title}{Reference Keyword}
%		Text Here
%	\end{prop}
% reference:
%	\ref{prop:Reference Keyword}
%%%%%%%%%%%%%%%%%%%%%%%%%%%%%%%%%%%%%%%%%%%%%%%%%%%%%%%%%%%%%%%%%%%%%%%%%%
\DeclareTcbTheorem[use counter from=dfn]{prop}{명제}{
	enhanced, breakable, left=4mm, right=4mm, top=4mm, bottom=2mm,
	colback=BurntOrange!5, coltitle=white, colframe=BurntOrange,
	before title={\,}, title={#1}, after title={\,}, fonttitle=\bfseries,
	attach boxed title to top left={xshift=4mm, yshift=-3.45mm},
	boxed title style={colback=BurntOrange, left=0mm,right=0mm, top=0mm, bottom=0mm,},
	separator sign none, description delimiters parenthesis
}{prop}

%%%%% TCBox - Example %%%%%%%%%%%%%%%%%%%%%%%%%%%%%%%%%%%%%%%%%%%%%%%%%%%%
% usage:
%	\begin{example}{Example Title}{Reference Keyword}
%		Text Here
%	\end{example}
% reference:
%	\ref{ex:Reference Keyword}
%%%%%%%%%%%%%%%%%%%%%%%%%%%%%%%%%%%%%%%%%%%%%%%%%%%%%%%%%%%%%%%%%%%%%%%%%%
\DeclareTcbTheorem[use counter from=dfn]{example}{예시}{
	enhanced, breakable, frame hidden, left=0mm, right=0mm, top=4mm, bottom=2mm,
	colback=white, coltitle=white, colframe=ProcessBlue,
	%borderline north={.5mm}{0pt}{ProcessBlue},
	borderline south={.5mm}{0pt}{ProcessBlue},
	before title={\,}, title={#1}, after title={\,}, fonttitle=\bfseries,
	attach boxed title to top left={yshift=-3.45mm},
	boxed title style={colback=ProcessBlue, sharp corners, left=0mm,right=0mm, top=0mm, bottom=0mm,},
	separator sign none, description delimiters parenthesis
}{ex}

%%%%% TCBox - Example Problem %%%%%%%%%%%%%%%%%%%%%%%%%%%%%%%%%%%%%%%%%%%%
% usage:
%	\begin{exc}{Example Title}{Reference Keyword}
%		Text Here
%	\end{exc}
% reference:
%	\ref{thm:Reference Keyword}
%%%%%%%%%%%%%%%%%%%%%%%%%%%%%%%%%%%%%%%%%%%%%%%%%%%%%%%%%%%%%%%%%%%%%%%%%%
\DeclareTcbTheorem[use counter from=dfn]{exc}{예제}{
	enhanced, breakable, frame hidden, left=0mm, right=0mm, top=4mm, bottom=2mm,
	colback=white, coltitle=white, colframe=NavyBlue,
	%borderline north={.5mm}{0pt}{NavyBlue},
	borderline south={.5mm}{0pt}{NavyBlue},
	before title={\,}, title={#1}, after title={\,}, fonttitle=\bfseries,
	attach boxed title to top left={yshift=-3.45mm},
	boxed title style={colback=NavyBlue, sharp corners, left=0mm,right=0mm, top=0mm, bottom=0mm,},
	separator sign none, description delimiters parenthesis
}{exc}

%%%%% TCBox - Proof %%%%%%%%%%%%%%%%%%%%%%%%%%%%%%%%%%%%%%%%%%%%%%%%%%%%%%
% usage:
%	\begin{proof}
%		Text Here
%	\end{Proof}
%%%%%%%%%%%%%%%%%%%%%%%%%%%%%%%%%%%%%%%%%%%%%%%%%%%%%%%%%%%%%%%%%%%%%%%%%%
\AtBeginDocument{\renewcommand\proofname{\bfseries 증명}} % You may edit here to change the proof name.
\tcolorboxenvironment{proof}{% `proof' from `amsthm'
	enhanced, blanker, breakable, toprule=.5mm, bottomrule=.5mm
}

%%%%% TCBox - Remark %%%%%%%%%%%%%%%%%%%%%%%%%%%%%%%%%%%%%%%%%%%%%%%%%%%%%
% usage:
%	\begin{remark}[Remark Title(default=참고)]
%		Text Here
%	\end{remark}
%%%%%%%%%%%%%%%%%%%%%%%%%%%%%%%%%%%%%%%%%%%%%%%%%%%%%%%%%%%%%%%%%%%%%%%%%%
\DeclareTColorBox{remark}{ O{참고} }{
	enhanced, breakable, blanker, coltitle=black, toprule=.5mm, bottomrule=.5mm,
	title={\bfseries #1}, after title={.~}, attach title to upper
}

%%%%% TCBox - Solution %%%%%%%%%%%%%%%%%%%%%%%%%%%%%%%%%%%%%%%%%%%%%%%%%%%
% usage:
%	\begin{sol}[Solution Title(default=풀이)]
%		Text Here
%	\end{sol}
%%%%%%%%%%%%%%%%%%%%%%%%%%%%%%%%%%%%%%%%%%%%%%%%%%%%%%%%%%%%%%%%%%%%%%%%%%
\DeclareTColorBox{sol}{ O{풀이} }{
	enhanced, frame hidden, breakable, left=0mm, right=0mm, top=2.5mm, bottom=2.5mm,
	borderline north={0.4mm}{.8pt}{black}, borderline south={0.4mm}{.8pt}{black},
	colback=white, coltitle=black,
	before title={\,}, title={#1}, after title={\,}, fonttitle=\bfseries,
	attach boxed title to top left={xshift=6mm, yshift=-3.5mm},
	boxed title style={left=0mm,right=0mm, top=0mm, bottom=0mm, sharp corners, frame hidden, colback = white,}
}

%%%%% TCBox - Hint %%%%%%%%%%%%%%%%%%%%%%%%%%%%%%%%%%%%%%%%%%%%%%%%%%%%%%%
% usage:
%	\begin{hint}[Hint Title(defalue=NONE)]
%		Text Here
%	\end{hint}
%%%%%%%%%%%%%%%%%%%%%%%%%%%%%%%%%%%%%%%%%%%%%%%%%%%%%%%%%%%%%%%%%%%%%%%%%%
\DeclareTColorBox{hint}{ O{}  }{
	enhanced, breakable, colframe=black, colback=white, sharp corners,
	IfValueT={#1}{before title={$\langle$}, title={#1}, after title={$\rangle$}, fonttitle=\bfseries},
	breakable, coltitle = black, top=2.5mm, left=2mm, right=2mm, bottom=2.5mm,  before skip=2mm,
	attach boxed title to top center={yshift=-3.45mm}, 
	boxed title style={left=0mm,right=0mm, top=0mm, bottom=0mm, sharp corners, frame hidden, colback = white},
}

%%%%% TCBox - Exercise Problem %%%%%%%%%%%%%%%%%%%%%%%%%%%%%%%%%%%%%%%%%%%
% usage:
%	\begin{prob}{Exercise Title}{Reference Keyword}
%		Text Here
%	\end{prob}
%%%%%%%%%%%%%%%%%%%%%%%%%%%%%%%%%%%%%%%%%%%%%%%%%%%%%%%%%%%%%%%%%%%%%%%%%%
\DeclareTcbTheorem[auto counter]{prob}{연습문제}{
	enhanced, breakable, frame hidden, left=0mm, right=0mm, top=4mm, bottom=2mm,
	colback=white, coltitle=white, colframe=RedOrange,
	borderline south={.5mm}{0pt}{RedOrange},
	before title={\,}, title={#1}, after title={\,}, fonttitle=\bfseries,
	attach boxed title to top left={yshift=-3.45mm},
	boxed title style={colback=RedOrange, sharp corners, left=0mm,right=0mm, top=0mm, bottom=0mm,},
	separator sign none, description delimiters parenthesis
}{prob}

%\pagenumbering{gobble}

\geometry{a4paper, total={6.4in, 10in}}
\title{고등학교 수학으로 증명하는 바젤 문제}

\begin{document}
\setstretch{1.3}
\maketitle
%\tableofcontents

\noindent 바젤문제(Basel Problem)는 1650년 이탈리아의 수학자 피에트로멩골리(Pietro Mengoli)가 제시한 문제이다. 문제의 이름은 이 문제를 오랫동안 공략하고 수학계에 널리 알린 야코프 베르누이가 근무했던 바젤대학교에서 유래했다.
\begin{hint}[바젤 문제]
	\centering 급수 $\displaystyle\sum_{n=1}^\infty\frac{1}{n^2}$의 값을 닫힌 형식으로 구하시오.
\end{hint}

수학자 레온하르트 오일러가 1735년에 이 급수가 $\pi^2/6$으로 수렴함을 불완전하게 증명하였다. 이후 1741년에 더욱 엄밀한 증명을 선보였으며, 약 100년 후 수학자 카를 바이어슈트라스가 바이어슈트라스 곱 정리로 증명의 타당성을 확보했다. 추후 이 문제에서 영감을 받아 급수를 리만 제타 함수로 일반화하였다. 리만 제타 함수와 관련된 가장 유명한 문제가 바로 리만 가설이다.

오일러의 증명 이외에도 수학자 오귀스탱 루이 코시의 초등적인 증명, 푸리에 급수를 이용한 증명 등의 풀이가 알려져 있으나, 이 칼럼에서는 고등학교 미적분 수준에서 증명하는 과정을 소개할 것이다.

\section{정적분 수열의 정의}
음이 아닌 정수 $n$에 대하여 두 수열 $\{a_n\}$과 $\{b_n\}$을 다음과 같이 정의하자.
\[
	a_n=\int_0^{\tfrac{\pi}{2}}\cos^{2n}x\,dx,\quad b_n=\int_0^{\tfrac{\pi}{2}}x^2\cos^{2n}x\,dx
\]
피적분함수와 적분 구간을 보면 자명하게 $a_n>0$, $b_n>0$이다.

\section{증명}
\begin{thm}{간단한 값}{2.1}
	\[
		a_0=\frac{\pi}{2},\quad b_0=\frac{\pi^3}{24}
	\]
\end{thm}
\begin{proof}
	\[
		a_0=\int_0^{\tfrac{\pi}{2}}1\,dx=\frac{\pi}{2},\quad b_0=\int_0^{\tfrac{\pi}{2}}x^2\,dx=\frac{\pi^3}{24}\qedhere
	\]
\end{proof}

\begin{thm}{점화식}{2.2}
	\[
		2na_n=(2n-1)a_{n-1}
	\]
\end{thm}

\begin{proof}
	\begin{align*}
		a_n&=\int_0^{\tfrac{\pi}{2}}\cos^{2n}x\,dx=\int_0^{\tfrac{\pi}{2}}\cos x\cos^{2n-1}x\,dx\\[3pt]
		&=\biggl[\sin x\cos^{2n-1}x\biggr]_0^{\tfrac{\pi}{2}}+\int_0^{\tfrac{\pi}{2}}(2n-1)\sin^2x\cos^{2n-2}x\,dx\\[3pt]
		&=\int_0^{\tfrac{\pi}{2}}(2n-1)(1-\cos^2x)\cos^{2n-2}x\,dx\\[3pt]
		&=(2n-1)\int_0^{\tfrac{\pi}{2}}(\cos^{2n-2}x-\cos^{2n}x)\,dx\\[3pt]
		&=(2n-1)(a_{n-1}-a_n)\qedhere
	\end{align*}
\end{proof}

\begin{thm}{정적분의 다른 형태}{2.3}
	\[
		a_n=2n\int_0^{\tfrac{\pi}{2}}x\sin x\cos^{2n-1}x\,dx
	\]
\end{thm}

\begin{proof}
	\begin{align*}
		a_n&=\int_0^{\tfrac{\pi}{2}}1\times\cos^{2n}x\,dx\\[3pt]
		&=\biggl[x\cos^{2n}x\biggr]_0^{\tfrac{\pi}{2}}-\int_0^{\tfrac{\pi}{2}}x(2n)(-\sin x)\cos^{2n-1}x\,dx\\[3pt]
		&=2n\int_0^{\tfrac{\pi}{2}}x\sin x\cos^{2n-1}x\,dx\qedhere
	\end{align*}
\end{proof}

\begin{thm}{두 수열의 관계}{2.4}
	\[
		\frac{a_n}{n^2}=\frac{2n-1}{n}b_{n-1}-2b_n
	\]
\end{thm}

\begin{proof}
	\begin{align*}
		\frac{a_n}{n}&=\int_0^{\tfrac{\pi}{2}}2x\sin x\cos^{2n-1}x\,dx\\[3pt]
		&=\biggl[x^2\sin x\cos^{2n-1}x\biggr]_0^{\tfrac{\pi}{2}}-\int_0^{\tfrac{\pi}{2}}x^2(\cos^{2n}x-(2n-1)(\cos^{2n-2}x-\cos^{2n}x))\,dx\\[3pt]
		&=-b_n+(2n-1)(b_{n-1}-b_n)\\
		&=(2n-1)b_{n-1}-2nb_n\qedhere
	\end{align*}
\end{proof}

\newpage
\begin{thm}{망원급수}{2.5}
	\[
		\frac{1}{n^2}=2\biggl(\frac{b_{n-1}}{a_{n-1}}-\frac{b_n}{a_n}\biggr)
	\]
\end{thm}

\begin{proof}
	정리 \ref{thm:2.4}에서 $\displaystyle\frac{1}{n^2}=\frac{2n-1}{na_n}b_{n-1}-\frac{2b_n}{a_n}$이고, 정리 \ref{thm:2.2}에서 $na_n=\dfrac{2n-1}{2}a_{n-1}$이므로
	\[
		\frac{1}{n^2}=\frac{2n-1}{\frac{2n-1}{2}a_{n-1}}b_{n-1}-\frac{2b_n}{a_n}=2\biggl(\frac{b_{n-1}}{a_{n-1}-\frac{b_n}{a_n}}\biggr).\qedhere
	\]
\end{proof}

\begin{thm}{부분합}{2.6}
	\[
		\sum_{k=1}^n\frac{1}{k^2}=\frac{\pi^2}{6}-\frac{2b_n}{a_n}
	\]
\end{thm}

\begin{proof}
	\[
		\displaystyle\sum_{k=1}^n\frac{1}{k^2}=\sum_{k=1}^n2\biggl(\frac{b_{k-1}}{a_{k-1}}-\frac{b_k}{a_k}\biggr)=2\biggl(\frac{b_0}{a_0}-\frac{b_k}{a_k}\biggr)=\frac{\pi^2}{6}-\frac{2b_n}{a_n}\qedhere
	\]
\end{proof}

\begin{thm}{부등식}{2.7}
	\[
		b_n\leq\int_0^{\tfrac{\pi}{2}}x^2\biggl(1-\frac{4x^2}{\pi^2}\biggr)^n\,dx
	\]
\end{thm}

\begin{proof}
	$0\leq x\leq\dfrac{\pi}{2}$일 때 $\sin x\geq\dfrac{2}{\pi}x\geq0$이므로
	\begin{align*}
		\sin x\geq\frac{2}{\pi}x~&\longrightarrow~\sin^2x\geq\biggl(\frac{2}{\pi}x\biggr)^2~\longrightarrow~1-\sin^2x\leq1-\biggl(\frac{2}{\pi}x\biggr)^2\\[3pt]
	&\longrightarrow~\cos^2x\leq1-\frac{4}{\pi^2}x^2~\longrightarrow~\cos^{2n}x\leq\biggl(1-\frac{4}{\pi^2}x^2\biggr)^n.
	\end{align*}
	그러므로 $x^2\geq0$으로부터
	\[
		b_n=\int_0^{\tfrac{\pi}{2}}x^2\cos^{2n}x\,dx\leq\int_0^{\tfrac{\pi}{2}}x^2\biggl(1-\frac{4x^2}{\pi^2}\biggr)^n\,dx.\qedhere
	\]
\end{proof}

\begin{thm}{다항함수의 정적분}{2.8}
	\[
		\int_0^{\tfrac{\pi}{2}}x^2\biggl(1-\frac{4x^2}{\pi^2}\biggr)^n\,dx=\frac{\pi^2}{8(n+1)}\int_0^{\tfrac{\pi}{2}}\biggl(1-\frac{4x^2}{\pi^2}\biggr)^{n+1}\,dx
	\]
\end{thm}

\begin{proof}
	\begin{align*}
		\int_0^{\tfrac{\pi}{2}}x^2\biggl(1-\frac{4x^2}{\pi^2}^n\,dx&=-\frac{\pi^2}{8}\int_0^{\tfrac{\pi}{2}}x\times\biggl(-\frac{8}{\pi^2}x\biggr)\biggl(1-\frac{4x^2}{\pi^2}\biggr)^n\,dx\\[3pt]
		&=-\frac{\pi^2}{8}\biggl(\biggl[\frac{x}{n+1}\biggl(1-\frac{4x^2}{\pi^2}\biggr)^{n+1}\biggr]_0^{\tfrac{\pi}{2}}-\int_0^{\tfrac{\pi}{2}}\frac{1}{n+1}\biggl(1-\frac{4x^2}{\pi^2}\biggr)^{n+1}\,dx\biggr)\\[3pt]
		&=\frac{\pi^2}{8(n+1)}\int_0^{\tfrac{\pi}{2}}\biggl(1-\frac{4x^2}{\pi^2}\biggr)^{n+1}\,dx\qedhere
	\end{align*}
\end{proof}

\begin{thm}{다항함수에서 삼각함수로}{2.9}
	\[
		b_n\leq\frac{\pi^3}{16(n+1)}\int_0^{\tfrac{\pi}{2}}\cos^{2n+3}x\,dx\leq\frac{\pi^3}{16(n+1)}a_n
	\]
\end{thm}

\begin{proof}
	정리 \ref{thm:2.7}과 정리 \ref{thm:2.8}로부터
	\[
		b_n\leq\frac{\pi^2}{8(n+1)}\int_0^{\tfrac{\pi}{2}}\biggl(1-\frac{4x^2}{\pi^2}^{n+1}\,dx
	\]
	이 성립한다. $x=\frac{\pi}{2}\sin t$로 치환하면 $\frac{4x^2}{\pi^2}=\sin^2t$이고 $\dfrac{dx}{dt}=\dfrac{\pi}{2}\cos t$이므로
	\[
		\int_0^{\tfrac{\pi}{2}}\biggl(1-\frac{4x^2}{\pi^2}\biggr)^{n+1}\,dx=\int_0^{\tfrac{\pi}{2}}\frac{\pi}{2}\cos t(1-\sin^2t)^{n+1}\,dt=\frac{\pi}{2}(\cos^2t)^{n+1}\cos t\,dt=\frac{\pi}{2}\int_0^{\tfrac{\pi}{2}}\cos^{2n+3}t\,dt.
	\]
	따라서
	\[
		b_n\leq\frac{\pi^2}{8(n+1)}\int_0^{\tfrac{\pi}{2}}\biggl(1-\frac{4x^2}{\pi^2}\biggr)^{n+1}\,dx=\frac{\pi^3}{16(n+1)}\int_0^{\tfrac{\pi}{2}}\cos^{2n+3}x\,dx
	\]
	가 성립하고, $0\leq x\leq\dfrac{\pi}{2}$일 때 $0\leq\cos x\leq1$이므로
	\[
		\int_0^{\tfrac{\pi}{2}}\cos^{2n+3}x\,dx\leq\int_0^{\tfrac{\pi}{2}}\cos^{2n}x\,dx=a_n
	\]
	이므로 주어진 부등식이 성립한다.
\end{proof}

\begin{thm}{부분합 부등식}{2.10}
	\[
		\frac{\pi^2}{6}-\frac{\pi^3}{8(n+1)}\leq\sum_{k=1}^n\frac{1}{k^2}<\frac{\pi^2}{6}
	\]
\end{thm}

\begin{proof}
	$a_n>0$, $b_n>0$이므로 정리 \ref{thm:2.6}으로부터
	\[
		\sum_{k=1}^n\frac{1}{k^2}=\frac{\pi^2}{6}-\frac{2b_n}{a_n}<\frac{\pi^2}{6}
	\]
	이고, 정리 \ref{thm:2.9}로부터
	\[
		\sum_{k=1}^n\frac{1}{k^2}=\frac{\pi^2}{6}-\frac{2b_n}{a_n}\geq\frac{\pi^2}{6}-\frac{2\cdot\frac{\pi^3}{16(n+1)}a_n}{a_n}=\frac{\pi^2}{6}-\frac{\pi^3}{8(n+1)}.\qedhere
	\]
\end{proof}

\begin{thm}{바젤 문제의 답}{2.11}
	\[
		\sum_{n=1}^\infty\frac{1}{n^2}=\frac{\pi^2}{6}
	\]
\end{thm}

\begin{proof}
	$\lim\limits_{n\to\infty}\biggl(\dfrac{\pi^2}{6}-\dfrac{\pi^3}{8(n+1)}\biggr)=\dfrac{\pi^2}{6}=\lim\limits_{n\to\infty}\dfrac{\pi^2}{6}$이므로 정리 \ref{thm:2.10}에 샌드위치 정리를 적용하면 $\displaystyle\sum_{n=1}^\infty\frac{1}{n^2}=\frac{\pi^2}{6}$.
\end{proof}

\begin{thebibliography}{9}
\bibitem{1}
	Daners, D. A short elementary proof of $\sum1/k^2=\pi^2/6$. Mathematics Magazine 2012, 85, 361–364. \url{https://doi.org/10.4169/math.mag.85.5.361}
\end{thebibliography}































\end{document}