\documentclass{article}
\usepackage[english]{babel}
\usepackage[utf8]{inputenc}
\usepackage[a4paper]{geometry}
\usepackage{mathtools}
\usepackage{amsthm, amsmath, amsfonts, amssymb}
\usepackage[]{mdframed}
\usepackage{hyperref}
\usepackage{multicol}
\usepackage[shortlabels]{enumitem}
\usepackage{float}
\usepackage{booktabs}
\usepackage[dvipsnames]{xcolor}
\usepackage{graphicx}
\usepackage{wrapfig}
\usepackage{tikz}
\usepackage{tikz-cd}
\usepackage[skins, vignette, raster, theorems, breakable]{tcolorbox}
\usepackage{pgfkeys}
\newcommand*\circled[1]{\tikz[baseline=(char.base)]{\node[shape=circle,draw,inner sep=1pt] (char) {#1};}}
\usepackage{textcomp, gensymb}
\usepackage{kotex}
\usepackage{setspace}
\usepackage{cancel}
\usepackage{diagbox}
\usepackage[no-math]{fontspec}

%%%%% font setting %%%%%%%%%%%%%%%%%%%%%%%%%%%%%%%%%%%%%%%%%%%%%%%%%%%%%%
\setmainfont{KoPub Dotum}[
	Extension = .ttf,
	UprightFont = * Light,
	BoldFont = * Bold,
	ItalicFont = * Light,
]
%%%%%%%%%%%%%%%%%%%%%%%%%%%%%%%%%%%%%%%%%%%%%%%%%%%%%%%%%%%%%%%%%%%%%%%%%

%%%%% author & date %%%%%%%%%%%%%%%%%%%%%%%%%%%%%%%%%%%%%%%%%%%%%%%%%%%%%
\author{}
\date{}
%%%%%%%%%%%%%%%%%%%%%%%%%%%%%%%%%%%%%%%%%%%%%%%%%%%%%%%%%%%%%%%%%%%%%%%%%

%%%%% TCBox - Definition %%%%%%%%%%%%%%%%%%%%%%%%%%%%%%%%%%%%%%%%%%%%%%%%
% usage:
%	\begin{dfn}{Definition Title}{Reference Keyword}
%		Text Here
%	\end{dfn}
% reference:
%	\ref{dfn:Reference Keyword}
%%%%%%%%%%%%%%%%%%%%%%%%%%%%%%%%%%%%%%%%%%%%%%%%%%%%%%%%%%%%%%%%%%%%%%%%%
\DeclareTcbTheorem[number within=section]{dfn}{정의}{
	enhanced, breakable, left=4mm, right=4mm, top=4mm, bottom=2mm,
	colback=Mahogany!5, coltitle=white, colframe=Mahogany,
	before title={\,}, title={#1}, after title={\,}, fonttitle=\bfseries,
	attach boxed title to top left={xshift=4mm, yshift=-3.45mm},
	boxed title style={colback=Mahogany, left=0mm,right=0mm, top=0mm, bottom=0mm},
	separator sign none, description delimiters parenthesis
}{dfn}

%%%%% TCBox - Theorem %%%%%%%%%%%%%%%%%%%%%%%%%%%%%%%%%%%%%%%%%%%%%%%%%%%
% usage:
%	\begin{thm}{Theorem Title}{Reference Keyword}
%		Text Here
%	\end{thm}
% reference:
%	\ref{thm:Reference Keyword}
%%%%%%%%%%%%%%%%%%%%%%%%%%%%%%%%%%%%%%%%%%%%%%%%%%%%%%%%%%%%%%%%%%%%%%%%%
\DeclareTcbTheorem[use counter from=dfn]{thm}{정리}{
	enhanced, breakable, left=4mm, right=4mm, top=4mm, bottom=2mm,
	colback=BurntOrange!5, coltitle=white, colframe=BurntOrange,
	before title={\,}, title={#1}, after title={\,}, fonttitle=\bfseries,
	attach boxed title to top left={xshift=4mm, yshift=-3.45mm},
	boxed title style={colback=BurntOrange,left=0mm,right=0mm, top=0mm, bottom=0mm,},
	separator sign none, description delimiters parenthesis
}{thm}

%%%%% TCBox - Lemma %%%%%%%%%%%%%%%%%%%%%%%%%%%%%%%%%%%%%%%%%%%%%%%%%%%%%
% usage:
%	\begin{lemma}{Lemma Title}{Reference Keyword}
%		Text Here
%	\end{lemma}
% reference:
%	\ref{lemma:Reference Keyword}
%%%%%%%%%%%%%%%%%%%%%%%%%%%%%%%%%%%%%%%%%%%%%%%%%%%%%%%%%%%%%%%%%%%%%%%%%
\DeclareTcbTheorem[use counter from=dfn]{lemma}{보조정리}{
	enhanced, breakable, left=4mm, right=4mm, top=4mm, bottom=2mm,
	colback=BurntOrange!5, coltitle=white, colframe=BurntOrange,
	before title={\,}, title={#1}, after title={\,}, fonttitle=\bfseries,
	attach boxed title to top left={xshift=4mm, yshift=-3.45mm},
	boxed title style={colback=BurntOrange, left=0mm,right=0mm, top=0mm, bottom=0mm},
	separator sign none, description delimiters parenthesis
}{lemma}

%%%%% TCBox - Proposition %%%%%%%%%%%%%%%%%%%%%%%%%%%%%%%%%%%%%%%%%%%%%%%%
% usage:
%	\begin{prop}{Proposition Title}{Reference Keyword}
%		Text Here
%	\end{prop}
% reference:
%	\ref{prop:Reference Keyword}
%%%%%%%%%%%%%%%%%%%%%%%%%%%%%%%%%%%%%%%%%%%%%%%%%%%%%%%%%%%%%%%%%%%%%%%%%%
\DeclareTcbTheorem[use counter from=dfn]{prop}{명제}{
	enhanced, breakable, left=4mm, right=4mm, top=4mm, bottom=2mm,
	colback=BurntOrange!5, coltitle=white, colframe=BurntOrange,
	before title={\,}, title={#1}, after title={\,}, fonttitle=\bfseries,
	attach boxed title to top left={xshift=4mm, yshift=-3.45mm},
	boxed title style={colback=BurntOrange, left=0mm,right=0mm, top=0mm, bottom=0mm,},
	separator sign none, description delimiters parenthesis
}{prop}

%%%%% TCBox - Example %%%%%%%%%%%%%%%%%%%%%%%%%%%%%%%%%%%%%%%%%%%%%%%%%%%%
% usage:
%	\begin{example}{Example Title}{Reference Keyword}
%		Text Here
%	\end{example}
% reference:
%	\ref{ex:Reference Keyword}
%%%%%%%%%%%%%%%%%%%%%%%%%%%%%%%%%%%%%%%%%%%%%%%%%%%%%%%%%%%%%%%%%%%%%%%%%%
\DeclareTcbTheorem[use counter from=dfn]{example}{예시}{
	enhanced, breakable, frame hidden, left=0mm, right=0mm, top=4mm, bottom=2mm,
	colback=white, coltitle=white, colframe=ProcessBlue,
	%borderline north={.5mm}{0pt}{ProcessBlue},
	borderline south={.5mm}{0pt}{ProcessBlue},
	before title={\,}, title={#1}, after title={\,}, fonttitle=\bfseries,
	attach boxed title to top left={yshift=-3.45mm},
	boxed title style={colback=ProcessBlue, sharp corners, left=0mm,right=0mm, top=0mm, bottom=0mm,},
	separator sign none, description delimiters parenthesis
}{ex}

%%%%% TCBox - Example Problem %%%%%%%%%%%%%%%%%%%%%%%%%%%%%%%%%%%%%%%%%%%%
% usage:
%	\begin{exc}{Example Title}{Reference Keyword}
%		Text Here
%	\end{exc}
% reference:
%	\ref{thm:Reference Keyword}
%%%%%%%%%%%%%%%%%%%%%%%%%%%%%%%%%%%%%%%%%%%%%%%%%%%%%%%%%%%%%%%%%%%%%%%%%%
\DeclareTcbTheorem[use counter from=dfn]{exc}{예제}{
	enhanced, breakable, frame hidden, left=0mm, right=0mm, top=4mm, bottom=2mm,
	colback=white, coltitle=white, colframe=NavyBlue,
	%borderline north={.5mm}{0pt}{NavyBlue},
	borderline south={.5mm}{0pt}{NavyBlue},
	before title={\,}, title={#1}, after title={\,}, fonttitle=\bfseries,
	attach boxed title to top left={yshift=-3.45mm},
	boxed title style={colback=NavyBlue, sharp corners, left=0mm,right=0mm, top=0mm, bottom=0mm,},
	separator sign none, description delimiters parenthesis
}{exc}

%%%%% TCBox - Proof %%%%%%%%%%%%%%%%%%%%%%%%%%%%%%%%%%%%%%%%%%%%%%%%%%%%%%
% usage:
%	\begin{proof}
%		Text Here
%	\end{Proof}
%%%%%%%%%%%%%%%%%%%%%%%%%%%%%%%%%%%%%%%%%%%%%%%%%%%%%%%%%%%%%%%%%%%%%%%%%%
\AtBeginDocument{\renewcommand\proofname{\bfseries 증명}} % You may edit here to change the proof name.
\tcolorboxenvironment{proof}{% `proof' from `amsthm'
	enhanced, blanker, breakable, toprule=.5mm, bottomrule=.5mm
}

%%%%% TCBox - Remark %%%%%%%%%%%%%%%%%%%%%%%%%%%%%%%%%%%%%%%%%%%%%%%%%%%%%
% usage:
%	\begin{remark}[Remark Title(default=참고)]
%		Text Here
%	\end{remark}
%%%%%%%%%%%%%%%%%%%%%%%%%%%%%%%%%%%%%%%%%%%%%%%%%%%%%%%%%%%%%%%%%%%%%%%%%%
\DeclareTColorBox{remark}{ O{참고} }{
	enhanced, breakable, blanker, coltitle=black, toprule=.5mm, bottomrule=.5mm,
	title={\bfseries #1}, after title={.~}, attach title to upper
}

%%%%% TCBox - Solution %%%%%%%%%%%%%%%%%%%%%%%%%%%%%%%%%%%%%%%%%%%%%%%%%%%
% usage:
%	\begin{sol}[Solution Title(default=풀이)]
%		Text Here
%	\end{sol}
%%%%%%%%%%%%%%%%%%%%%%%%%%%%%%%%%%%%%%%%%%%%%%%%%%%%%%%%%%%%%%%%%%%%%%%%%%
\DeclareTColorBox{sol}{ O{풀이} }{
	enhanced, frame hidden, breakable, left=0mm, right=0mm, top=2.5mm, bottom=2.5mm,
	borderline north={0.4mm}{.8pt}{black}, borderline south={0.4mm}{.8pt}{black},
	colback=white, coltitle=black,
	before title={\,}, title={#1}, after title={\,}, fonttitle=\bfseries,
	attach boxed title to top left={xshift=6mm, yshift=-3.5mm},
	boxed title style={left=0mm,right=0mm, top=0mm, bottom=0mm, sharp corners, frame hidden, colback = white,}
}

%%%%% TCBox - Hint %%%%%%%%%%%%%%%%%%%%%%%%%%%%%%%%%%%%%%%%%%%%%%%%%%%%%%%
% usage:
%	\begin{hint}[Hint Title(defalue=NONE)]
%		Text Here
%	\end{hint}
%%%%%%%%%%%%%%%%%%%%%%%%%%%%%%%%%%%%%%%%%%%%%%%%%%%%%%%%%%%%%%%%%%%%%%%%%%
\DeclareTColorBox{hint}{ O{}  }{
	enhanced, breakable, colframe=black, colback=white, sharp corners,
	IfValueT={#1}{before title={$\langle$}, title={#1}, after title={$\rangle$}, fonttitle=\bfseries},
	breakable, coltitle = black, top=2.5mm, left=2mm, right=2mm, bottom=2.5mm,  before skip=2mm,
	attach boxed title to top center={yshift=-3.45mm}, 
	boxed title style={left=0mm,right=0mm, top=0mm, bottom=0mm, sharp corners, frame hidden, colback = white},
}

%%%%% TCBox - Exercise Problem %%%%%%%%%%%%%%%%%%%%%%%%%%%%%%%%%%%%%%%%%%%
% usage:
%	\begin{prob}{Exercise Title}{Reference Keyword}
%		Text Here
%	\end{prob}
%%%%%%%%%%%%%%%%%%%%%%%%%%%%%%%%%%%%%%%%%%%%%%%%%%%%%%%%%%%%%%%%%%%%%%%%%%
\DeclareTcbTheorem[auto counter]{prob}{연습문제}{
	enhanced, breakable, frame hidden, left=0mm, right=0mm, top=4mm, bottom=2mm,
	colback=white, coltitle=white, colframe=RedOrange,
	borderline south={.5mm}{0pt}{RedOrange},
	before title={\,}, title={#1}, after title={\,}, fonttitle=\bfseries,
	attach boxed title to top left={yshift=-3.45mm},
	boxed title style={colback=RedOrange, sharp corners, left=0mm,right=0mm, top=0mm, bottom=0mm,},
	separator sign none, description delimiters parenthesis
}{prob}

%\pagenumbering{gobble}

\geometry{a4paper, total={6.4in, 10in}}
\title{자연수/정수 조건의 처리}

\begin{document}
\setstretch{1.3}
\maketitle

\noindent 예로부터 수능에서는 미지수가 자연수 또는 정수로 제한된 조건을 제시하는 문제를 출제하였다. 자연수 또는 정수는 실수와 달리 연속성을 갖지 않기에 문제에 해당 조건이 있으면 복잡한 경우 나누기가 필요하다. 복잡한 경우의 수 시험에서 한 문제에 소비되는 시간이 필연적으로 증가함을 의미하며, 경우 분류를 최적화하지 않으면 시간을 허비하게 된다. 이 칼럼에서는 자연수와 정수의 근본적인 성질로 경우 분류 및 조건 처리의 기준과 시작점을 제시할 것이다.

\section{소인수분해}
모든 자연수는 산술의 기본 정리에 따라 소수들의 곱으로 표현하는 방법, 즉 소인수분해가 유일하게 존재한다.

\begin{thm}{산술의 기본 정리}{1.1}
	모든 자연수는 유일한 소인수분해를 갖는다.
\end{thm}

소인수분해로 얻은 소수의 곱 형태를 이용하거나, 소인수에 주목하는 방식으로 조건의 처리를 시도할 수 있다.

\begin{exc}{2017년 3월 학력평가 나형 21번}{1.2}
	자연수 $m$에 대하여 집합 $A_m$을
	\[
		A_m=\biggl\{(a,b)\mid 2^a=\frac{m}{b},a,b\text{는 자연수}\biggr\}
	\]
	라 할 때, $\langle$보기$\rangle$에서 옳은 것만을 있는 대로 고른 것은? [4점]
	\begin{hint}[보기]
		\begin{itemize}
			\item[ㄱ.] $A_4=\{(1,2),(2,1\}$
			\item[ㄴ.] 자연수 $k$에 대하여 $m=2^k$이면 $n(A_m)=k$이다.
			\item[ㄷ.] $n(A_m)=1$이 되도록 하는 두 자리 자연수 $m$의 개수는 $23$이다.
		\end{itemize}
	\end{hint}
\end{exc}

\begin{sol}
	ㄴ과 ㄷ에서 집합 $A_m$의 원소의 개수 $n(A_m)$에 대한 명제를 제시하고 있으므로 먼저 $n(A_m)$을 구해보자. 집합 $A_m$의 정의로부터
	\[
		2^a=\frac{m}{b}~\longrightarrow~m=2^ab
	\]
	이다. $m$의 소인수분해에서 나타나는 $2$의 개수(지수)를 $a_m$이라 하고, $b_m=m/2^{a_m}$이라 하자. 즉 $m$의 소인수분해는 $m=2^{a_m}b_m$이다. 따라서 집합 $A_m$에서 가능한 $a$의 값은 $1,\cdots,a_m$으로 모두 $a_m$개이다.
	\begin{itemize}
		\item[ㄱ.] $4=2^2$이므로 $A_4=\{(1,2),(2,1)\}$이다. (참)
		\item[ㄴ.] 위의 논의에 따라 $m=2^k$이면 $n(A_m)=k$이다. (참)
		\item[ㄷ.] $n(A_m)=1$이 되려면 $m=2\times(\text{홀수})$가 되어야 하므로 가능한 홀수는 $5,7,9,\cdots,49$로 총 $23$개이다. (참)\qed
	\end{itemize}
\end{sol}

\section{경우 분류 (1) -- 약수/배수 관계}
조건으로 제시된 수의 범위와 상관없이, 문제를 푸는 일반적인 방법은 미지수나 문자 사이의 관계식을 찾는 것이다. 이때 $(\text{정수})\times(\text{정수})=(\text{정수})$의 형태로 식을 변형하면 정수 사이의 약수/배수 관계를 이용할 수 있다. 이는 정수 조건의 부정방정식을 푸는 기본 사고방식이 된다.

\begin{exc}{2018년 6월 고2 학력평가 나형 20번}{2.1}
	첫째항이 $-36$이고 공차가 $d$인 등차수열 $\{a_n\}$이 있다. 다음 조건을 만족하는 모든 자연수 $d$의 값의 합은? [4점]
	\begin{hint}
		\begin{itemize}
			\item[(가)] 모든 자연수 $n$에 대하여 $a_n\ne 0$이다.
			\item[(나)] $\displaystyle\sum_{k=1}^ma_k=0$인 자연수 $m$이 존재한다.
		\end{itemize}
	\end{hint}
\end{exc}

\begin{sol}
	수열 $\{a_n\}$의 일반항은 $a_n=-36+(n-1)d$이다.
	\begin{itemize}
		\item[(가)] 모든 자연수 $n$에 대하여 $(n-1)d\ne 36$이다. 즉 $d$는 $36$의 약수가 아니다.
		\item[(나)] $\displaystyle\sum_{k=1}^ma_k=\frac{m}{2}(a+1+a_m)=\frac{m}{2}(-72+(m-1)d)$에서 $(m-1)d=72$이다. 즉 $d$는 $72$의 약수이다.
	\end{itemize}
	$d$는 $36$의 약수가 아니면서 $72$의 약수이므로 모든 $d$의 합은 $8+24+72=104$이다.\qed
\end{sol}

\begin{remark}[참고]
	수열은 정의역이 자연수 전체의 집합인 함수이다. 따라서 이 문제에서 자연수 제약 걸린 문자는 $d,n,m$의 세 가지이다. 자연수 조건이 주어진 경우, 수열의 항 번호도 자연수임을 적극적으로 활용하여 부정방정식을 세워야 한다.
\end{remark}

\section{경우 분류 (2) -- 나눗셈의 나머지}
나눗셈에서 나머지를 고려하는 것은 정수를 탐구하는 가장 기본적인 자세이다. 특히 나머지로 가능한 값 사이의 연산을 통해 불필요한 경우 제거에 도움이 된다. 자연수 $n$으로 나누었을 때 등장할 수 있는 나머지는 $0,1,\cdots,n-1$의 $n$가지이다. 이때 이 수들이 나머지임을 명시하기 위해 가로선을 그어 $\overline{0},\overline{1},\cdots,\overline{n-1}$이라 하고, 집합 $\mathbb Z_n=\{\overline{0},\overline{1},\cdots,\overline{n-1}\}$로 정의하자. 즉 $\mathbb Z_n$은 $n$으로 나눈 나머지의 집합이 된다. 이때 $\mathbb Z_n$의 두 원소 $\overline{x},\overline{y}$ 사이에 다음 두 연산 $\dot+,\dot\times$를 정의하자.
\[
	\overline{x}\dot+\overline{y}=\overline{(x+y)\text{ mod }n},\quad\overline{x}\dot\times\overline{y}=\overline{xy\text{ mod }n}
\]
여기서 $m\text{ mod }n$은 $m$을 $n$으로 나눈 나머지이다. 예를 들어 $\mathbb Z_4=\{\overline{0},\overline{1},\overline{2},\overline{3}\}$에 대하여
\[
	\overline{2}\dot+\overline{3}=\overline{(2+3)\text{ mod }4}=\overline{1},\quad\overline{2}\dot\times\overline{3}=\overline{2\times 3\text{ mod }4}=\overline{2}
\]
이다. 이 방식으로 나머지끼리 먼저 계산하면 직접 계산하기 전에 빠르게 경우 분류 및 모순 판단이 가능하다.

\newpage
\begin{exc}{2019년 10월 학력평가 나형 29번}{3.1}
	첫째항이 짝수인 수열 $\{a_n\}$은 모든 자연수 $n$에 대하여
	\[
		a_{n+1}=\begin{cases}
			a_n+3 & (a_n\text{이 홀수인 경우})\\[3pt]
			\dfrac{a_n}{2} & (a_n\text{이 짝수인 경우}
		\end{cases}
	\]
	를 만족시킨다. $a_5=5$일 때, 수열 $\{a_n\}$의 첫째항이 될 수 있는 모든 수의 합을 구하시오. [4점]	
\end{exc}

\begin{sol}
	$a_n$이 홀수이면 $a_{n+1}=a_n+3$은 짝수이다. 반대로 $a_n$이 짝수이면 $a_{n+1}$은 홀수와 짝수 모두 가능하다. $a_5=5$는 홀수이므로 $a_4$은 짝수이고 $a_4=10$이다. $a_4$는 짝수이므로 $a_3$은 두 가지 경우를 고려해야 한다.
	\begin{enumerate}[(i)]
		\item $a_3$이 홀수이면 $a_3=7$이다. 같은 논리로 $a_2$는 짝수이며 $a_2=14$이다. $a_1$은 짝수이므로 $a_1=28$이다.
		\item $a_3$이 짝수이면 $a_3=20$이다. 이제 $a_2$에도 두 가지 경우가 있다. $a_2$가 홀수이면 $a_2=17$, $a_1=34$이다. $a_2$가 짝수이면 $a_2=40$, $a_1=80$이다.
	\end{enumerate}
	(i), (ii)로부터 모든 $a_1$의 값의 합은 $28+34+80=142$이다.\qed
\end{sol}

\begin{remark}[참고]
	풀이에서는 글로 길게 썼지만, 실제 풀이에서는 간단한 수형도를 그려 $a_1$을 빠르게 구하면 된다.
\end{remark}

\section{기타 전략}
몇몇 문제에서는 매우 유명한 아이디어가 등장하기도 한다. 대표적으로 수의 진법 변환을 활용한 식이 등장한다.

\begin{exc}{2021학년도 수능특강}{4.1}
	수열 $\{a_n\}$은 $a_1=1$이고, 모든 자연수 $n$에 대하여
	\[
		\begin{cases}
			a_{2n}=a_n\\
			a_{2n+1}=a_n+1
		\end{cases}
	\]
	을 만족시킨다. $100$ 이하의 자연수 $k$에 대하여 $a_k=2$인 모든 자연수 $k$의 개수는?
\end{exc}

\begin{sol}
	$a_n$은 $n$을 이진수로 변환하였을 때 등장하는 $1$의 개수이다. (Why?) $a_k=2$임은 곧 이진수 자연수에서 등장하는 $1$의 개수가 단 두 개임의 의미한다. 이때 $2^6=64<100$이고 $2^7=128>100$이므로 고려해야 할 이진수의 자릿수는 $2^0$에서 $2^6$까지 총 $7$개이다. 따라서 $7$개의 자리 중에서 $1$이 들어갈 두 자리를 고르는 경우의 수는 $21$이다.\qed
\end{sol}

\newpage
\section*{연습문제}
\begin{prob}{2022학년도 수능특강}{1}
	자연수 $m$에 대하여 집합 $A_m$을
	\[
		A_m=\{(a,b)\mid m=a\log_2b\text{이고},a,b\text{는 자연수}\}
	\]
	라 할 때, $\langle$보기$\rangle$에서 옳은 것만을 있는 대로 고른 것은?
	\begin{hint}[보기]
		\begin{itemize}
			\item[ㄱ.] $A_2=\{(1,4),(2,2)\}$
			\item[ㄴ.] 두 자연수 $p,q$에 대하여 $n(A_{pq})=n(A_p)\times n(A_q)$이다.
			\item[ㄷ.] $n(A_m)=4$를 만족시키는 $30$ 이하의 모든 자연수 $m$의 개수는 $9$이다.
		\end{itemize}
	\end{hint}
\end{prob}

\begin{sol}
	ㄴ과 ㄷ에서 집합 $A_m$의 원소의 개수에 대한 명제를 제시하고 있으므로 먼저 $n(A_m)$을 구해보자. 집합 $A_m$의 정의에서
	\[
		m=a\log_2b~\longrightarrow~b=2^{m/a}
	\]
	이고 $a,b$ 모두 자연수이므로 $m/a$는 자연수이다. 따라서 $a$는 $m$의 약수이고 $n(A_m)$은 $m$의 약수의 개수이다.
	\begin{itemize}
		\item[ㄱ.] $a$로 가능한 값은 $1,2$이고, 이에 따른 $b$의 값은 각각 $4,2$이다. (참)
		\item[ㄴ.] $(pq\text{의 약수의 개수})=(p\text{의 약수의 개수})\times(q\text{의 약수의 개수})$는 일반적으로 성립하지 않는다. (거짓)
		\item[ㄷ.] 약수의 개수가 $4$가 되기 위해서는 자연수 $m$이 서로 다른 두 소수 $p_1,p_2$의 곱 $m=p_1p_2$이거나 소수 $p$의 세제곱 $m=p^3$여야 한다. $30$ 이하의 자연수 중에서 그러한 수는 $9$개이다. (참)\qed
	\end{itemize}
\end{sol}

\newpage
\begin{prob}{2022학년도 9월 모의평가 공통 13번}{2}
	첫째항이 $-45$이고 공차가 $d$인 등차수열 $\{a_n\}$이 다음 조건을 만족시키도록 하는 모든 자연수 $d$의 값의 합은? [4점]
	\begin{hint}
		\begin{itemize}
			\item[(가)] $\vert a_m\vert=\vert a_{m+3}\vert$인 자연수 $m$이 존재한다.
			\item[(나)] 모든 자연수 $n$에 대하여 $\displaystyle\sum_{k=1}^na_k>-100$이다.
		\end{itemize}
	\end{hint}
\end{prob}

\begin{sol}
	증가 또는 감소만 하는 등차수열의 경향에 따라 (가)로부터 $a_m=-a_{m+3}$이므로, 이를 정리하면 $(2m+1)d=90$이다. 이때 $2m+1$은 $3$ 이상의 홀수이므로 다음과 같이 경우를 나눌 수 있다.
	\begin{table}[H]
		\centering
		\begin{tabular}{c|ccccc}
			\toprule
			$2m+1$ & $3$ & $5$ & $9$ & $15$ & $45$ \\
			\midrule
			$d$ & $30$ & $18$ & $10$ & $6$ & $2$ \\
			\midrule
			$(m,d)$ & $(1,30)$ & $(2,18)$ & $(4,10)$ & $(7,6)$ & $(22,2)$  \\
			\bottomrule
		\end{tabular}
	\end{table}
	$a_m+a_{m+3}=0$로부터 $a_{m+1}+a_{m+2}=0$이고 $a_{m+1}<a_{m+2}$이므로 $a_{m+1}<0$이고 $a_{m+2}>0$이다. 따라서 수열 $\{a_n\}$은 $(m+1)$항까지만 음수이고 이후로는 모두 양수이다. 따라서 $\displaystyle\sum_{k=1}^na_k$의 최솟값은
	\[
		\sum_{k=1}^{m+1}a_k=\frac{(m+1)(a_1+a_{m+1}}{2}=\frac{(m+1)(md-90)}{2}=\frac{(m+1)(md-(2m+1)d)}{2}=-\frac{(m+1)^2d}{2}
	\]
	이다. 이 값이 $-100$보다 커야 하므로 $(m+1)^2d<200$이다. (가)의 각 경우를 확인하자.
	\begin{table}[H]
		\centering
		\begin{tabular}{c|ccccc}
			\toprule
			$(m,d)$ & $(1,30)$ & $(2,18)$ & $(4,10)$ & $(7,6)$ & $(22,2)$  \\
			\midrule
			$(m+1)^2d$ & $120<200$ & $168<200$ & $250\geq200$ & $364\geq200$ & $1058\geq200$ \\
			\bottomrule
		\end{tabular}
	\end{table}
	따라서 가능한 모든 자연수 $d$의 합은 $30+18=48$이다.\qed
\end{sol}

\begin{remark}[참고]
	(나)에서 $(m+1)^2d<200$을 이끌어내는 대신에 (가)에서 구한 $(m,d)$를 직접 대입하고 음수인 항을 나열하여 계산하는 것이 더 편하다. $90$ 대신엣 $(2m+1)d$를 대입하는 행위는 다소 발상적이기 때문이다. 그럼에도 이 풀이를 제시하는 이유는 자연수/정수 조건이 주어진 경우에 과감한 계산과 식 세우기를 강조하기 위함이다.
\end{remark}

\newpage
\begin{prob}{2021년 4월 학력평가 공통 21번}{3}
	첫째항이 자연수인 수열 $\{a_n\}$이 모든 자연수 $n$에 대하여
	\[
		a_{n+1}=\begin{cases}
			a_n-2 & (a_n\geq0) \\
			a_n+5 & (a_n<0)
		\end{cases}
	\]
	을 만족시킨다. $a_{15}<0$이 되도록 하는 $a_1$의 최솟값을 구하시오. [4점]
\end{prob}

\begin{sol}
	$a_1$은 자연수, 즉 양수이므로 $2$를 빼는 행위를 자연수 범위에서 지속하지 못하는 경우에 수열에서 음수가 등장한다. $2$를 계속 빼는 연산의 결과는 $a_1$의 홀짝성에 좌우된다. (자연수의 나눗셈은 뺄셈의 반복임에 주목하라.) 따라서 $a_1$의 홀짝성에 따라 경우를 나누어 항의 부호가 바뀌는 지점을 관찰하자.
	\begin{enumerate}[(i)]
		\item $a_1=2k$ ($k$는 자연수)인 경우: $a_1$부터 $a_k$까지 모두 양수이고, $a_{k+1}=0$, $a_{k+2}=-2$이다.
		\item $a_1=2k-1$ ($k$는 자연수)인 경우: $a_1$부터 $a_k$까지 모두 양수이고, $a_{k+1}=-1$, $a_{k+2}=4$이다. $(k+2)$항이 짝수이므로 (i)과 같은 논리로 $a_{k+4}=0$, $a_{k+5}=-2$이다.
	\end{enumerate}
	(i), (ii)로부터 $a_1$의 홀짝성에 관계없이 $-2$가 등장함을 알 수 있다. $-2$부터 몇 개의 항을 더 나열해보면 다음과 같다.
	\[
		-2~\rightarrow~3~\rightarrow~1~\rightarrow~-1~\rightarrow~4~\rightarrow~2~\rightarrow~0~\rightarrow~-2
	\]
	수열 $\{a_n\}$은 음수가 등장하면서부터 주기성을 갖는다. 이제 각 경우에 대하여 수열에서 음수인 항을 먼저 등장하는 순서대로 나열하면 다음과 같다. (화살표 위의 숫자는 항 번호의 규칙을 나타낸다.)
	\begin{enumerate}[(i)]
		\item $a_1=2k$ ($k$는 자연수)인 경우:
			\[
				a_{k+2}=-2~\xrightarrow{+3}~a_{k+5}=-1~\xrightarrow{+4}~a_{k+9}=-2~\xrightarrow{+3}~a_{p+12}=-1~\xrightarrow{+4}~a_{p+16}=-2~\xrightarrow{}~\cdots
			\]
		\item $a_1=2k-1$ ($k$는 자연수)인 경우:
			\[
				a_{k+1}=-1~\xrightarrow{+4}~a_{k+5}=-2~\xrightarrow{+3}~a_{k+8}=-1~\xrightarrow{+4}~a_{p+12}=-2~\xrightarrow{+3}~a_{p+15}=-1~\xrightarrow{}~\cdots
			\]
	\end{enumerate}
	그러므로 $a_{15}<0$이면서 $a_1$이 최소이려면 $k+12=15$, 즉 $k=3$이어야 하며, 이때 $a_1$의 최솟값은 $5$이다.\qed
\end{sol}

\begin{remark}[참고]
	공식 해설에서는 $a_1=1$부터 직집 대입하여 확인한다. 이는 답이 작기 때문에 유효한 전략이며, 약간의 문제 변형으로 무용지물이 된다. 물론 실전에서는 여러 $a_1$의 값을 실험해보면서 주기성을 발견하고 이를 일반화해야 한다.	
\end{remark}

\newpage
\begin{prob}{2023학년도 수능 공통 15번}{4}
	모든 항이 자연수이고 다음 조건을 만족시키는 모든 수열 $\{a_n\}$에 대하여 $a_9$의 최댓값과 최솟값을 각각 $M,m$이라 할 때, $M+m$의 값은? [4점]
	\begin{hint}
		\begin{itemize}
			\item[(가)] $a_7=40$
			\item[(나)] 모든 자연수 $n$에 대하여
				\[
					a_{n+2}=\begin{cases}
						a_{n+1}+a_n & (a_{n+1}\text{이 }3\text{의 배수가 아닌 경우}) \\[3pt]
						\dfrac{1}{3}a_{n+1} & (a_{n+1}\text{이 }3\text{의 배수인 경우})
					\end{cases}
				\]
				이다.
		\end{itemize}
	\end{hint}
\end{prob}

\begin{sol}
	수열 $\{a_n\}$의 한 항은 이전 항의 $3$의 배수 여부에 따라 정의된다. 따라서 $a_n$을 $3$을 나눈 나머지를 기준으로 경우를 분류하자. $a_n$을 $3$으로 나눈 나머지를 $b_n$이라 하자. 아래는 수열 $\{b_n\}$의 귀납적 정의이다.
	\begin{enumerate}[(i)]
		\item $b_{n+1}=1,2$이면 $b_{n+2}$는 $b_{n+1}+b_n$을 $3$으로 나눈 나머지이다.
		\item $b_{n+1}=0$이면 $b_{n+2}$는 $a_{n+1}/3$을 $3$으로 나눈 나머지이다.
	\end{enumerate}
	(i), (ii)에서 다음을 얻는다.
	\begin{enumerate}[(a)]
		\item 수열 $\{a_n\}$의 어떤 항이 $3$으로 나누어떨어지지 않으면 그 앞의 항도 찾아야 한다.
		\item 수열 $\{a_n\}$의 어떤 항이 $3$으로 나누어떨어지면 다음 항을 찾을 수 있다.
	\end{enumerate}
	지금까지의 결론을 바탕으로 문제를 풀면 아래의 표로 끝난다. 표의 숫자들은 $(b_n)_{a_n}$의 형태로 나열되어 있다. 예를 들어 $n=7$일 때, 표에서 $1_40$은 $a_7=40$, $b_7=1$임을 뜻한다.
	\begin{table}[H]
		\centering
		\begin{tabular}{c|cccccc}
			\toprule
			$n$ & $4$ & $5$ & $6$ & $7$ & $8$ & $9$ \\
			\midrule
			경우 1 ($b_6=0$) & & & $0_{120}$ & $1_{40}$ & $1_{160}$ & $2_{200}$ \\
			\midrule
			경우 2 ($b_6=1$) & & $0_{30}$ & $1_{10}$ & $1_{40}$ & $2_{50}$ & $0_{90}$ \\
			\midrule
			경우 3 ($b_6=2$) & $0_{24}$ & $2_8$ & $2_{32}$ & $1_{40}$ & $0_{72}$ & $1_{24}$ \\
			\bottomrule
		\end{tabular}
	\end{table} 
	$M=200$, $m=24$이므로 $M+m=224$이다.\qed
\end{sol}

\newpage
\begin{prob}{2022년 10월 학력평가 공통 15번}{5}
	수열 $\{a_n\}$의 첫째항부터 제$n$항까지의 합을 $S_n$이라 하자. 두 자연수 $p,q$에 대하여 $S_n=pn^2-36n+q$일 때, $S_n$이 다음 조건을 만족시키도록 하는 $p$의 최솟값을 $p_1$이라 하자.
	\begin{hint}
		임의의 두 자연수 $i,j$에 대하여 $i\ne j$이면 $S_i\ne S_j$이다.
	\end{hint}
	$p=p_1$일 때, $\vert a_k\vert<a_1$을 만족시키는 자연수 $k$의 개수가 $3$이 되도록 하는 모든 $q$의 값의 합은? [4점]
\end{prob}

\begin{sol}
	주어진 조건에 의해 $S_i=S_j$은 $i=j$ 이외의 경우에는 성립할 수 없다. $S_i-S_j$를 계산하면
	\[
		S_i-S_j=(pi^2-36i+q)-(pj^2-36j+q)=(i-j)(p(i+j)-36)
	\]
	이다. 이 값은 $i=j$ 이외의 조건에서는 $0$이 될 수 없으므로 $p_1=5$이다. $p=p_1$일 때, 수열 $\{a_n\}$은 다음과 같다.
	\[
		\begin{cases}
			a_1=S_1=q-31 \\
			a_n=S_n-S_{n-1}=10n-41 & (n\geq2)
		\end{cases}
	\]
	항의 절댓값이 특정 값보다 작은 상황을 고려하기 위해 $\{a_n\}$에서 부호가 변하는 지점을 관찰하자.
	\[
		a_2=-21,\quad a_3=-11,\quad a_4=-1,\quad a_5=9,\quad a_6=19
	\]
	에서 $\vert a_k\vert<a_1=q-31$을 만족하는 $3$개의 $k$는 절댓값이 가장 작은 세 항의 번호인 $3,4,5$가 되어야 한다. 따라서 $11<q-31\geq19$에서 $42<q<50$이고 이를 만족하는 모든 자연수 $q$의 합은 $43+\cdots+50=372$이다.\qed
\end{sol}

\newpage
\begin{prob}{2022학년도 사관학교 공통 15번}{6}
	다음 조건을 만족시키는 모든 수열 $\{a_n\}$에 대하여 $a_1$의 최솟값을 $m$이라 하자.
	\begin{hint}
		\begin{itemize}
			\item[(가)] 수열 $\{a_n\}$의 모든 항은 정수이다.
			\item[(나)] 모든 자연수 $n$에 대하여
				\[
					a_{2n}=a_3\times a_n+1,\quad a_{2n+1}=2a_n-a_2
				\]
				이다.
		\end{itemize}
	\end{hint}
	$a_1=m$인 수열 $\{a_n\}$에 대하여 $a_9$의 값은? [4점]
\end{prob}

\begin{sol}
	두 식에 $n=1$을 대입하면 $a_2=a_3\times a_1+1$, $a_3=2a_1-a_2$이다. $a_3$과 $a_1$에 대하여 정리하면
	\[
		a_3=\frac{2a_1-1}{a_1+1}=2-\frac{3}{a_1+1}
	\]
	이고, $a_3$은 정수이므로 $a_1+1$은 (부호를 고려한) $3$의 약수이다. 따라서 $m=-4$이다. $a_1=m$일 때, $a_3=3$, $a_2=-11$이므로 $a_9=2a_4-a_2=2(a_3\times a_2+1)-a_2=-53$이다.\qed
\end{sol}

\begin{remark}[참고]
	$(\text{정수})\times(\text{정수})=(\text{정수})$의 형태로 변형해도 되지만, 유리식으로 변형하면 최솟값을 구하기 쉽다.
\end{remark}

\newpage
\begin{prob}{2023학년도 6월 모의평가 공통 15번}{7}
	자연수 $k$에 대하여 다음 조건을 만족시키는 수열 $\{a_n\}$이 있다.
	\begin{hint}
		$a_1=0$이고, 모든 자연수 $n$에 대하여
		\[
			a_{n+1}=\begin{cases}
				a_n+\dfrac{1}{k+1} & (a_n\leq0) \\[9pt]
				a_n-\dfrac{1}{k} & (a_n>0)
			\end{cases}
		\]
		이다.
	\end{hint}
	$a_{22}=0$이 되도록 하는 모든 $k$의 값의 합은? [4점]
\end{prob}

\begin{sol}
	임의의 자연수 $n$에 대하여 $a_n$은 $a_1=0$에 $\frac{1}{k+1}$을 더하거나 $\frac{1}{k}$을 빼는 행위를 $(n-1)$번 시행하여 얻는다. 이때 주어진 $n$에 대하여 $a_n$을 얻기까지 $\frac{1}{k}$을 빼는 행위를 $p_n$회 시행했다고 하면, $\frac{1}{k+1}$을 더하는 행위는 $(n-1-p_n)$회 시행한 것이 된다. 즉 다음 식이 성립한다.
	\[
		a_n=\frac{n-1-p_n}{k+1}-\frac{p_n}{k}
	\]
	$a_{22}=0$으로부터
	\[
		a_{22}=\frac{21-p_{22}}{k+1}-\frac{p_{22}}{k}=0~\longrightarrow~p_{22}=\frac{21k}{2k+1}
	\]
	이다. $p_{22}$는 자연수이므로 $2k+1$은 $21k$의 약수여야 한다. $k$와 $2k+1$은 서로소이므로 $2k+1$은 $21$의 약수이다. 따라서 가능한 자연수 $k$는 $1,3,10$이고 그 합은 $14$이다.\qed
\end{sol}

\newpage
\begin{prob}{2006학년도 사관학교 나형 10번}{8}
	수열의 합 $\displaystyle\sum_{k=1}^n2^k$의 값이 $65$의 배수가 되도록 하는 자연수 $n$의 최솟값은? [3점]
\end{prob}

\begin{sol}
	주어진 합은 $2(2^n-1)$이고 $2$와 $65$는 서로소이므로 $2^n-1$이 $65$의 배수가 되어야 한다. $65=5\times 13$이므로 $2^n$을 $5$와 $13$으로 나눈 나머지를 관찰하자. $2^n$을 $5$로 나눈 나머지는 $2,4,3,1$이 순서대로 반복된다. 따라서 $n$은 $4$의 배수가 되어야 한다. 자연수 $m$에 대하여 $n=4m$이라 하자. $2^{4m}=16^m=(13+3)^m$이므로 $2^{4m}$을 $13$으로 나눈 나머지는 $3^m$을 $13$으로 나눈 나머지와 같다. $3^m$을 $13$으로 나눈 나머지는 $3,9,1$이 순서대로 반복된다. 따라서 $m$은 $3$의 배수이다. 그러므로 $n$은 $12$의 배수여야 하므로 최솟값은 $12$이다.\qed
\end{sol}

\newpage
\begin{prob}{2021학년도 6월 모의평가 가형 21번}{9}
	수열 $\{a_n\}$의 일반항은
	\[
		a_n=\log_2\sqrt{\frac{2(n+1)}{n+2}}
	\]	
	이다. $\displaystyle\sum_{k=1}^ma^k$의 값이 $100$ 이하의 자연수가 되도록 하는 모든 자연수 $m$의 값의 합은? [4점]
\end{prob}

\begin{sol}
	$a_n=\dfrac{1}{2}(1+\log_2(n+1)-\log_2(n+2))$이므로
	\begin{align*}
		\sum_{k=1}^ma_k&=\frac{1}{2}\sum_{k=1}^m(1+\log_2(n+1)-\log_2(n+2))\\[3pt]
		&=\frac{1}{2}(m+\log_22-\log_2(m+2))\\[3pt]
		&=\frac{1}{2}(m+1-\log_2(m+2))
	\end{align*}
	이다. 이 값이 자연수가 되려면 $m+2$는 $2$의 거듭제곱이어야 하고, $m+1-\log_2(m+2)$는 짝수여야 한다. $m+1$은 홀수이므로 $m+2$는 $2$의 홀수 거듭제곱이다. 또 $m+2\geq 2^8=256$이면 값이 $100$을 초과하게 된다. 따라서 가능한 자연수 $m$은 다음과 같다.
	\begin{table}[H]
		\centering
		\begin{tabular}{c|ccc}
			\toprule
			$m+2$ & $2^3=8$ & $2^5=32$ & $2^7=128$ \\
			\midrule
			$m$ & $6$ & $30$ & $126$ \\
			\midrule
			$\sum_{k=1}^ma_k$ & $2$ & $13$ & $60$ \\
			\bottomrule
		\end{tabular}
	\end{table}
	구하는 모든 자연수 $m$의 합은 $162$이다.\qed
\end{sol}

\newpage
\begin{prob}{2020학년도 4월 학력평가 가형 21번}{10}
	자연수 $k$에 대하여 집합 $A_k$를
	\[
		A_k=\left\{\sin\frac{2(m-1)}{k}\pi\,\middle\vert\,m\text{은 자연수}\right\}
	\]
	라 할 때, $\langle$보기$\rangle$에서 옳은 것만을 있는 대로 고른 것은? [4점]
	\begin{hint}[보기]
		\begin{itemize}
			\item[ㄱ.] $A_3=\left\{-\dfrac{\sqrt3}{2},0,\dfrac{\sqrt3}{2}\right\}$
			\item[ㄴ.] $1$이 집합 $A_k$의 원소가 되도록 하는 두 자리 자연수 $k$의 개수는 $22$이다.
			\item[ㄷ.] $n(A_k)=11$을 만족시키는 모든 $k$의 값의 합은 $33$이다.
		\end{itemize}
	\end{hint}
\end{prob}

\begin{sol}
	$\sin\frac{2(m-1)}{k}\pi=\sin\frac{2\pi}{k}(m-1)$이므로 집합 $A_k$는 좌표평면 위의 단위원 $x^2+y^2=1$에 내접하면서 점 $(1,0)$을 한 꼭짓점으로 하는 정$k$각형의 꼭짓점의 $y$좌표의 집합이다. 이 정$k$각형은 $x$축에 대하여 대칭이다.
	\begin{itemize}
		\item[ㄱ.] $\sin 0=0$, $\sin\dfrac{2\pi}{3}=\dfrac{\sqrt3}{2}$, $\sin\dfrac{4\pi}{3}=-\dfrac{\sqrt3}{2}$ (참)
		\item[ㄴ.] $1$이 $A_k$의 원소가 되기 위새ㅓ는 꼭짓점 중 하나가 $(0,1)$이어야 하므로 $k$는 $4$의 배수여야 한다. 따라서 가능한 두 자리 자연수는 $12$부터 $96$까지 $22$개이다. (참)
		\item[ㄷ.] $k$가 $4$의 배수이거나 $4$로 나눈 나머지가 $2$인 수이면 정$k$각형은 $y$축에 대하여 대칭이다. 이때 $k$가 $4$의 배수인 경우에는 점 $(-1,0)$, $(0,1)$, $(0,-1)$을 모두 꼭짓점이므로, 꼭짓점의 $y$좌표가 $11$가지이려면 $0<y<1$인 부분에 $8$개의 꼭짓점이 있어야 한다. 대칭성에 의해 $-1<y<0$인 부분에도 $8$개의 꼭짓점이 있어야 하므로 이때의 $k$는 $4+8+8=20$이다. $k$를 $4$로 나눈 나머지가 $2$이면 점 $(-1,0)$은 꼭짓점이고 점 $(0,1)$, $(0,-1)$은 꼭짓점이 아니므로 $0<y<1$인 부분과 $-1<y<0$인 부분에 각각 $10$개의 꼭짓점이 있어야 한다. 따라서 이때의 $k$는 $2+10+10=22$이다. 이외의 경우에는 $y$축 대칭이 아니므로 정확히 $11$개의 꼭짓점이 존재해야 한다. 그러므로 모든 $k$의 값의 합은 $20+22+11=53$이다. (거짓)\qed
\end{itemize}
\end{sol}

\newpage
\begin{prob}{2019학년도 6월 고2 학력평가 나형 21번}{11}
	음이 아닌 세 정수 $a$, $b$, $n$에 대하여
	\[
		(a^2+b^2+2ab-4)\cos\frac{n}{4}\pi+(b^2+ab+2)\tan\frac{2n+1}{4}\pi=0
	\]
	일 때, $a+b+\sin^2\dfrac{n}{8}\pi$의 값은? (단, $a\geq b$) [4점]
\end{prob}

\begin{sol}
	자연수 $n$에 대하여 $\cos\frac{n}{4}\pi$와 $\tan\frac{2n+1}{4}\pi$는 각각 $8$, $2$의 주기를 갖는다. 이를 음이 아닌 정수 $k$에 대하여 다음 표와 같이 정리할 수 있다.
	\begin{table}[H]
		\centering
		\begin{tabular}{c|cccccccc}
			\toprule
			$n$ & $8k$ & $8k+1$ & $8k+2$ & $8k+3$ & $8k+4$ & $8k+5$ & $8k+6$ & $8k+7$ \\ 
			\midrule
			$\cos\frac{n}{4}$ & $1$ & $\frac{\sqrt2}{2}$ & $0$ & $-\frac{\sqrt2}{2}$ & $-1$ & $-\frac{\sqrt2}{2}$ & $0$ & $\frac{\sqrt2}{2}$ \\ \addlinespace[3pt]
			\midrule
			$\tan\frac{2n+1}{4}\pi$ & $1$ & $-1$ & $1$ & $-1$ & $1$ & $-1$ & $1$ & $-1$ \\ \addlinespace[3pt]
			\bottomrule
		\end{tabular}
	\end{table}
	\begin{enumerate}[(i)]
		\item $n$이 $8k+1$, $8k+3$, $8k+5$, $8k+7$ 중 하나인 경우: 이 경우 식에 무리수인 $\sqrt2$가 포함되므로 계수에 해당하는 $a^2+b^2+2ab-4$와 $b^2+ab+2$는 모두 $0$이어야 한다. 특히
			\[
				b^2+ab+2=0~\longrightarrow~b(a+b)=-2<0
			\]
			에서 $b$는 음이 아닌 정수이므로 $b>0$이다. 따라서 $a+b<0$에서 $a<-b<0$이므로 이는 $a$가 음이 아닌 정수임에 모순이다.
		\item $n$이 $8k+2$, $8k+6$ 중 하나인 경우: $b^2+ab+2=0$이므로 (i)과 같은 논리로 이 식을 만족하는 음이 아닌 정수 $a$, $b$는 존재하지 않는다.
		\item $n$ $8k$인 경우: $a^2+b^2+2ab-4=-b^2-ab-2$에서 $a^2+3ab+2b^2=2$, $(a+b)(a+2b)=2$이고 $a+b<a+2b$이므로 $a+b=1$, $a+2b=2$에서 $(a,b)=(0,1)$이다. 이는 $a\geq b$를 만족하지 않는다.
		\item $n$이 $8k+4$인 경우: $a^2+b^2+2ab-4=b^2+ab+2$에서 $a(a+b)=6$이고 $a<a+b$이므로 $(a,b)=(1,5),(2,1)$이고, 여기서 $a\geq b$를 만족하는 것은 $(2,1)$이다.
	\end{enumerate}
	(i)--(iv)에서 $a=2$, $b=1$이고, $\sin^2\frac{n}{8}\pi=\sin^2\frac{8k+4}{8}\pi=1$이다. 따라서 구하는 값은 $2+1+1=4$이다.\qed
\end{sol}

\begin{remark}[참고]
	정수 조건과 삼각함수의 주기성을 동시에 생각하면 나눗셈의 나머지를 통한 경우 분류라는 아이디어는 필연적이다. 이후 무리수의 상등을 이용하여 절반의 경우의 수를 동시에 다룬다. 또한 모든 경우에서 $(\text{정수})\times(\text{정수})=(\text{정수})$의 형태로 식을 변형하고 인수의 대소를 비교하여 약수/배수 관계를 적용한다.
\end{remark}

\newpage
\begin{prob}{2022학년도 6월 모의평가 공통 21번}{12}
	다음 조건을 만족시키는 최고차항의 계수가 $1$인 이차함수 $f(x)$가 존재하도록 하는 모든 자연수 $n$의 값의 합을 구하시오. [4점]
	\begin{hint}
		\begin{itemize}
			\item[(가)] $x$에 대한 방정식 $(x^n-64)f(x)=0$은 서로 다른 두 실근을 갖고, 각각의 실근은 중근이다.
			\item[(나)] 함수 $f(x)$의 최솟값은 음의 정수이다.
		\end{itemize}
	\end{hint}
\end{prob}

\begin{sol}
	$n$이 홀수일 때 방정식 $x^n-64=0$은 유일한 실근 $\sqrt[n]{64}$를 갖는다. 따라서 이차식 $f(x)$를 곱하도라도 방정식 $(x^n-64)f(x)=0$은 서로 다른 두 개의 중근을 가질 수 없다. $n$이 짝수이면서 방정식 $x^n-64=0$의 실근은 $\sqrt[n]{64}$, $-\sqrt[n]{64}$로 두 개이다. 따라서 (가)를 만족시키기 위해서는
	\[
		f(x)=(x-\sqrt[n]{64})(x+\sqrt[n]{64})=x^2-2^{\frac{12}{n}}
	\]
	이어야 한다. 함수 $f(x)$의 최솟값은 $-2^{\frac{12}{n}}$이다. 이 값이 음의 정수이므로 $n$이 $12$의 약수가 되어야 한다. (가)에서 $n$이 짝수이므로 가능한 $n$은 $12$의 약수 중에서 짝수인 $2,4,6,12$이고, 합은 $24$이다.\qed
\end{sol}

\newpage
\begin{prob}{2019년 6월 고2 학력평가 가형 21번}{13}
	자연수 $n$에 대하여 $f(n)$이 다음과 같다.
	\[
		f(x)=\begin{cases}
			\sqrt[4]{9\times 2^{n+1}} & ($n$\text{이 홀수}) \\
			\sqrt[4]{4\times 3^n} & ($n$\text{이 짝수})
		\end{cases}
	\]
	$10$ 이하의 두 자연수 $p$, $q$에 대하여 $f(p)\times f(q)$가 자연수가 되도록 하는 모든 순서쌍 $(p,q)$의 개수는? [4점]
\end{prob}

\begin{sol}
	$p$, $q$의 홀짝성에 따라 경우를 나누자.
	\begin{enumerate}[(i)]
		\item $p$와 $q$ 모두 홀수인 경우: $f(p)\times f(q)=\sqrt[4]{9\times 2^{p+1}}\times\sqrt[4]{9\times 2^{q+1}}=3\sqrt[4]{2^{p+q+2}}$이므로 $p+q+2$는 $4$의 배수여야 한다.
			\begin{table}[H]
				\centering
				\begin{tabular}{c|l}
					\toprule
					$p+q$ & $(p,q)$ \\
					\midrule
					$2$ & $(1,1)$ \\
					\midrule
					$6$ & $(1,5),(3,3),(5,1)$ \\
					\midrule
					$10$ & $(1,9),(3,7),(5,5),(7,3),(9,1)$ \\
					\midrule
					$14$ & $(5,9),(7,7),(9,5)$ \\
					\midrule
					$18$ & $(9,9)$ \\
					\bottomrule
				\end{tabular}
			\end{table}
			그러므로 순서쌍 $(p,q)$의 개수는 $13$이다.
		\item $p$가 홀수 $q$가 짝수인 경우: $f(p)\times f(q)=\sqrt[4]{9\times 2^{p+1}}\times\sqrt[4]{4\times 3^q}=\sqrt[4]{2^{p+3}\times 3^{q+2}}$이므로 $p+3$과 $q+4$ 모두 $4$의 배수여야 한다. $p$로 가능한 수는 $1,5,9$이고 $q$로 가능한 수는 $2,6,10$이므로 순서쌍 $(p,q)$의 개수는 $3\times3=9$이다.
		\item $p$가 짝수, $q$가 홀수인 경우: (ii)와 같은 논리로 순서쌍 $(p,q)$의 개수는 $9$이다.
		\item $p$와 $q$ 모두 짝수인 경우: $f(p)\times f(q)=\sqrt[4]{4\times 3^p}\times\sqrt[4]{4\times 3^q}=2\sqrt[4]{3^{p+q}}$이므로 $p+q$는 $4$의 배수여야 한다. 이 경우는 $(i)$의 경우와 거의 동일하므로 순서쌍 $(p,q)$의 개수는 $13$이다. ((i)의 순서쌍의 각 성분에 모두 $1$을 더하면 $(iv)$에 해당하는 순서쌍을 얻는다.)
	\end{enumerate}
	(i)--(iv)로부터 구하는 순서쌍의 개수는 $44$이다.\qed
\end{sol}

\newpage
\begin{prob}{2018학년도 사관학교 나형 28번}{14}
	$2$ 이상의 자연수 $n$에 대하여 $n^{\frac{4}{k}}$의 값이 자연수가 되도록 하는 자연수 $k$의 개수를 $f(n)$이라 하자. 예를 들어 $f(6)=3$이다. $f(n)=8$을 만족시키는 $n$의 최솟값을 구하시오. [4점]
\end{prob}

\begin{sol}
	$n$의 소인수분해를 $p_1^{q_1}p_2^{q_2}\cdots p_{n_k}^{q_{n_k}}$라 하자. 지수에 해당하는 $q_1,\cdots, q_{n_k}$의 최대공약수를 $d_n$이라 하자. $n^\frac{4}{k}$의 값이 자연수가 되기 위해서는 $k$가 $4d_n$의 약수여야 한다. 약수의 개수가 $8$인 최소의 자연수는 $2^3\times3=4\times 6=24$이므로 $d_n$이 $6$인 최소의 $n$은 $2^6=64$이다.\qed
\end{sol}

\newpage
\begin{prob}{2023학년도 수능특강}{15}
	$p>0$, $q<0$인 두 정수 $p$, $q$와 모든 자연수 $n$에 대하여 수열 $\{a_n\}$이
	\[
		a_1=30,\quad a_{n+1}=a_n+2pn+q
	\]
	를 만족시킨다. 두 부등식 $a_3>0$, $a_4<0$이 모두 성립하도록 하는 정수 $p$, $q$의 모든 순서쌍 $(p,q)$의 개수를 구하시오.
\end{prob}

\begin{sol}
	먼저 $a_3$과 $a_4$를 구하자.
	\[
		a_1=30~\longrightarrow~a_2=30+2p+q~\longrightarrow~a_3=30+6p+2q~\longrightarrow~a_4=30+12p+3q
	\]
	$a_3>0$에서 $-q<15+3p$이고, $a_4<0$에서 $-q>10+4p$이다. 두 부등식을 연립하면
	\[
		10+4p<-q<15+3p~\longrightarrow~10+4p<15+3p~\longrightarrow~p<5
	\]
	이므로 $p=1,2,3,4$에 대하여 가능한 순서쌍을 찾으면 다음과 같다.
	\begin{align*}
		&p=1~\longrightarrow~14<-q<18~\longrightarrow~(1,-15),(1,-16),(1,-17)\\
		&p=2~\longrightarrow~18<-q<21~\longrightarrow~(2,-19),(2,-20)\\
		&p=3~\longrightarrow~22<-q<24~\longrightarrow~(3,-23)\\
		&p=4~\longrightarrow~26<-q<27~\longrightarrow~\text{가능한 }q\text{가 존재하지 않는다.}
	\end{align*}
	따라서 구하는 순서쌍의 개수는 $6$이다.\qed
\end{sol}

\newpage
\begin{prob}{2024학년도 수능특강}{16}
	수열 $\{a_n\}$이 모든 자연수 $n$에 대하여
	\[
		a_{n+1}=\begin{cases}
			n+a_n & (a_n<n) \\
			a_n-p & (a_n\geq n)
		\end{cases}
	\]
	을 만족시킨다. 수열 $\{a_n\}$이 다음 조건을 만족시키도록 하는 모든 $p$의 값의 합을 구하시오.
	\begin{hint}
		\begin{itemize}
			\item[(가)] $p$는 $10$ 이하의 자연수이다.
			\item[(나)] $a_m=0$, $a_{m+4}=0$인 자연수 $m$이 존재한다.
		\end{itemize}
	\end{hint}
\end{prob}

\begin{sol}
	먼저 $a_{m+3}$까지는 무리없이 구할 수 있다.
	\[
		a_m=0~\longrightarrow~a_{m+1}=m~\longrightarrow~a_{m+2}=2m+1~\longrightarrow~a_{m+3}=2m+1-p
	\]
	$a_{m+3}$으로부터 $a_{m+4}$를 구하려고 시도하면 경우를 분류해야 함을 알 수 있다.
	\begin{enumerate}[(i)]
		\item $2m+1-p<m+3$, 즉 $m<p+2$인 경우: $a_{m+4}=3m+4-p=0$에서 $p=3m+4$이다. $m<p+2=3m+6$에서 $m>-3$이다. 따라서 가능한 자연수 $m$, $p$의 순서쌍 $(m,p)$는 $(1,7),(2,10)$이다.
		\item $2m+1-p\geq m+3$, 즉 $m\geq p+2$인 경우: $a_{m+4}=2m+1-2p=0$에서 $p=m+\frac{1}{2}$이므로 이는 $p$가 자연수임에 모순이다.
	\end{enumerate}
	(i), (ii)로부터 모든 $p$의 값의 합은 $17$이다.\qed
\end{sol}



























\end{document}